\documentclass{beamer}
% Language/Font package
\usepackage[utf8]{inputenc}
\usepackage[francais]{babel}
\usepackage[T1]{fontenc}

% Base package
\usepackage{graphicx}
\usepackage{color}
\usepackage{listings}
\usepackage{hyperref}

% Configuration
\definecolor{ligthYellow}{RGB}{255,255,229}

\lstset{
language=bash,
numbers=left,
numberstyle=\small,
numbersep=8pt,
basicstyle=\small\ttfamily,
tabsize=4,
showspaces=false,
showstringspaces=false,
stringstyle=\color{red}\ttfamily,
commentstyle=\color{green}\ttfamily,
breaklines=true,
frame = single,
framexleftmargin=15pt,
backgroundcolor=\color{ligthYellow}}

% Theme https://github.com/matze/mtheme
\usetheme[progressbar=frametitle]{metropolis}

% Title
\title{Présentation Git}
\subtitle{Un outil de collaboration puissant}
\date{\today}
\author{Denis Pettens}
\institute{Louvain-li-Nux}
\titlegraphic{\hfill\includegraphics[height=2cm]{img/logo.png}}

\begin{document}
  \maketitle

  \begin{frame}{Table des matières}
      \setbeamertemplate{section in toc}[sections numbered]
      \tableofcontents[hideallsubsections]
  \end{frame}

  \section{Introduction}
  \begin{frame}{Monsieur, c'est quoi Git?}
      \begin{itemize}
          \item Git est un logiciel de gestion de versions décentralisé
      \end{itemize}
  \end{frame}
  \begin{frame}{Pourquoi l'utiliser?}

  \end{frame}
  \begin{frame}{GitKraken}

  \end{frame}
  \section{Instalation et configuration}
\begin{frame}[fragile]
     \frametitle{Installer Git}
      \textbf{Ubuntu}\\
      \begin{lstlisting}
      test
      \end{lstlisting}
      \textbf{OS X}\\
          \url{https://sourceforge.net/projects/git-osx-installer/}
      \textbf{Windows}\\
          \url{https://git-for-windows.github.io/}
\end{frame}
  \begin{frame}{Installer GitKraken}

  \end{frame}
  \begin{frame}{Configuration de base}

  \end{frame}
  \section{Premier pas avec Git}
  \begin{frame}{git init}

  \end{frame}
  \begin{frame}{git add}

  \end{frame}
  \begin{frame}{git commit}

  \end{frame}
  \section{Manipuler l'historique}
  \section{Les branches}
  \section{Le travail en groupe}


  \plain{Questions?}
\end{document}
